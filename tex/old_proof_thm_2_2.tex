\begin{proof}\label{app:proof_directional_derivative}
    By definition of the directional derivative,
    \begin{align}
      H'(z; \delta)
      = & \lim_{h \downarrow 0} \frac{H(z + h \delta) - H(z)}{h} \nonumber \\
      = &
      \lim_{h \downarrow 0} \frac{1}{h}
      \left(
        |z + h \delta| \smashoperator{\sum_{j \in C(z + h \delta)}} \lambda_{(j)^-_{z + h \delta}}
        + \smashoperator{\sum_{j \notin C(z + h\delta)}} |\beta_j| \lambda_{(j)^-_{z + h \delta}}
        - |z| \smashoperator{\sum_{j \in C(z)}} \lambda_{(j)^-_z}
        - \smashoperator{\sum_{j \notin C(z)}} |\beta_j| \lambda_{(j)^-_z}
      \right) \nonumber                                                          \\
      = & \lim_{h \downarrow 0}
      \frac{1}{h}
      \left(
        |z + h \delta| \smashoperator{\sum_{j \in C(z + {\varepsilon_c}\delta)}} \lambda_{(j)^-_{z + {\varepsilon_c}\delta}}
        + \smashoperator{\sum_{j \notin C(z + {\varepsilon_c}\delta)}} |\beta_j| \lambda_{(j)^-_{z + {\varepsilon_c}\delta}}
        - |z| \smashoperator{\sum_{j \in C(z)}} \lambda_{(j)^-_{z}}
        - \smashoperator{\sum_{j \notin C(z)}} |\beta_j| \lambda_{(j)^-_{z}}
      \right)
      \label{eq:directional-derivative-sl1}
    \end{align}
    We have the following cases to consider: \(|z| \in c^{\setminus k}\),
    \(|z| = 0\), and \(|z| \notin \{0\} \cup c^{\setminus k}\).


    Let us first consider the case
    $z \neq 0$, $z \neq c_i$ for all $i \neq k$.
    Here since \(C(z + \delta h) = C(z)\) for $h$ small enough,
    %  has the directional derivative
    % \[
    %   H'(z; \delta) = \delta \sign(z) \smashoperator{\sum_{j \in C(z)}} \lambda_{(j)^-_{z}}.
    % \]

    Let us now treat the case \(|z| = c_i^{\setminus k}\), note that we have,
    as a result of the definition of
    \({\varepsilon_c}\), the following identities:
    \begin{align*}
      C(c_i^{\setminus k} + {\varepsilon_c}\delta)            & \subseteq C(c_i^{\setminus k}),                                                                                                                                 \\
      \tilde{\mathcal{C}}_i                           & = \widebar{C(c_i^{\setminus k} + {\varepsilon_c} \delta)} \cap C(c_i^{\setminus k}),                                                                                    \\
      C(c_i^{\setminus k})                                    & = \tilde{C}_i \cup \big(C(c_i^{\setminus k} + {\varepsilon_c}\delta) \cap C(c_i^{\setminus k})\big) = \tilde{\mathcal{C}}_i \cup C(c_i^{\setminus k} + {\varepsilon_c} \delta), \\
      \widebar{C(c_i^{\setminus k} + {\varepsilon_c} \delta)} & = \widebar{C(c_i^{\setminus k})} \cup \tilde{\mathcal{C}}_i.
    \end{align*}
    Using this, we can rewrite \eqref{eq:directional-derivative-sl1} as
    \begin{equation*}
      \label{eq:directional-derivative-simplified}
      H'(z; \delta)
      = \lim_{h \downarrow 0} \frac{1}{h}
      \left(
        \splitfrac{
        |z + h\delta|\smashoperator{\sum_{j \in C(z + {\varepsilon_c}\delta)}} \lambda_{(j)^-_{z + {\varepsilon_c}\delta}}
        + |c_i^{\setminus k}|\smashoperator{\sum_{j \in \tilde{C}_i}} \lambda_{(j)^-_{z + {\varepsilon_c}\delta}}
        + \smashoperator{\sum_{j \in \widebar{C(z)}}} |\beta_j|\lambda_{(j)^-_{z + {\varepsilon_c}\delta}}
        }{%
        - |z| \smashoperator{\sum_{j \in \tilde{\mathcal{C}}_i}} \lambda_{(j)^-_{z}}
        - |c_i^{\setminus k}| \smashoperator{\sum_{j \in C(z + {\varepsilon_c}\delta)}} \lambda_{(j)^-_{z}}
        - \smashoperator{\sum_{j \in \widebar{C(z)}}} |\beta_j| \lambda_{(j)^-_{z}}
        }
      \right).
    \end{equation*}
    Next, observe that \(\lambda_{(j)^-_{z + {\varepsilon_c}\delta}} =
    \lambda_{(j)^-_{z}}\) for all \(j \in \widebar{C(z)}\) and consequently
    \[
      \smashoperator{\sum_{j \in \widebar{C(z)}}} |\beta_j|\lambda_{(j)^-_{z + {\varepsilon_c}\delta}} =
      \smashoperator{\sum_{j \in \widebar{C(z)}}} |\beta_j|\lambda_{(j)^-_{z}}.
    \]
    Moreover, note that, since \(z = \pm c_i\), there exists a permutation corresponding to
    \(\lambda_{(j)^-_{z}}\) such that
    \[
      |z|\smashoperator{\sum_{j \in \tilde{C}_i}} \lambda_{(j)^-_{z + {\varepsilon_c}\delta}}
      = |c^{\setminus k}_i| \smashoperator{\sum_{j \in \tilde{C}_i}} \lambda_{(j)^-_{z}}
    \]
    and consequently
    \begin{equation}
      \label{eq:directional-derivative-simplified-again}
      H'(z; \delta)
      = \lim_{h \downarrow 0} \frac{1}{h}
      \left(
        |z + h\delta|\smashoperator{\sum_{j \in C(z + {\varepsilon_c}\delta)}} \lambda_{(j)^-_{z + {\varepsilon_c}\delta}}
        - |c_i^{\setminus k}| \smashoperator{\sum_{j \in C(z + {\varepsilon_c}\delta)}} \lambda_{(j)^-_{z}}.
      \right).
    \end{equation}
    Now, since \(c_i^{\setminus k} + h \delta > 0\) and \(-c_i^{\setminus k} + h \delta < 0\) in the limit as
    \(h\) goes to \(0\) for \(c_i \neq 0\), we have
    \[
      \lim_{h\downarrow 0} |-c_i + h \delta|
      = \lim_{h\downarrow 0}\big( |c_i^{\setminus k}| -h \delta\big)
      \quad\text{and}\quad
      \lim_{h\downarrow 0} |c_i^{\setminus k} + h \delta|
      = \lim_{h\downarrow 0}(|c_i^{\setminus k}| + h \delta)
    \]
    which means that
    \begin{align*}
      H'(z; \delta)
       & = \lim_{h \downarrow 0} \frac{1}{h}
      \left(
        \big(|z| + \sign(z)h\delta\big)\smashoperator{\sum_{j \in C(z + {\varepsilon_c}\delta)}} \lambda_{(j)^-_{z + {\varepsilon_c}\delta}}
        - |c_i| \smashoperator{\sum_{j \in C(z + {\varepsilon_c}\delta)}} \lambda_{(j)^-_{z}}.
      \right)                                                                                                                  \\
       & = \lim_{h \downarrow 0} \frac{1}{h}
      \sign(z)h\delta\smashoperator{\sum_{j \in C(z + {\varepsilon_c}\delta)}} \lambda_{(j)^-_{z + {\varepsilon_c}\delta}}     \\
       & = \sign(z)\delta\smashoperator{\sum_{j \in C(z + {\varepsilon_c}\delta)}} \lambda_{(j)^-_{z + {\varepsilon_c}\delta}} \\
    \end{align*}

    For the case when \(z=0\) there are two sub-cases to consider. First, when \(c_m^{\setminus k} = 0\) then
    \eqref{eq:directional-derivative-simplified-again} is simply
    \begin{equation*}
      \label{eq:directional-derivative-zerocase}
      H'(0; \delta)
      = \lim_{h \downarrow 0} \frac{1}{h}
      |h\delta|\smashoperator{\sum_{j \in C({\varepsilon_c}\delta)}} \lambda_{(j)^-_{{\varepsilon_c}\delta}}
      = \smashoperator{\sum_{j \in C({\varepsilon_c}\delta)}} \lambda_{(j)^-_{{\varepsilon_c}\delta}}
    \end{equation*}
    since \(|\delta| = 1\) by definition.

    The other sub-case is \(c_m^{\setminus k} \neq 0\). Here
    \eqref{eq:directional-derivative-sl1} reduces to
    \begin{equation*}
      H'(0; \delta) = \lim_{h \downarrow 0}
      \frac{1}{h}
      \left(
        |h \delta| \smashoperator{\sum_{j \in C({\varepsilon_c}\delta)}} \lambda_{(j)^-_{{\varepsilon_c}\delta}}
        + \smashoperator{\sum_{j \notin C({\varepsilon_c}\delta)}} |\beta_j| \lambda_{(j)^-_{{\varepsilon_c}\delta}}
        - \smashoperator{\sum_{j \notin C(0)}} |\beta_j| \lambda_{(j)^-_{0}}
      \right).
    \end{equation*}
    In this case, we have \(C(0) = C({\varepsilon_c}\delta)\) since \(c_m^{\setminus k}
    \neq 0\), and therefore
    \[
      \smashoperator{\sum_{j \notin C({\varepsilon_c}\delta)}} |\beta_j| \lambda_{(j)^-_{{\varepsilon_c}\delta}}
      = \smashoperator{\sum_{j \notin C(0)}} |\beta_j| \lambda_{(j)^-_{0}},
    \]
    which means that
    \begin{equation*}
      H'(0; \delta) = \lim_{h \downarrow 0}
      \frac{1}{h}
      |h \delta| \smashoperator{\sum_{j \in C({\varepsilon_c}\delta)}} \lambda_{(j)^-_{{\varepsilon_c}\delta}}
      = |\delta|\smashoperator{\sum_{j \in C({\varepsilon_c}\delta)}} \lambda_{(j)^-_{{\varepsilon_c}\delta}}
      = \smashoperator{\sum_{j \in C(0)}} \lambda_{(j)^-_{0}}.
    \end{equation*}
  \end{proof}

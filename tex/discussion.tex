\section{DISCUSSION}\label{sec:discussion}
%%%%%%%%%%%%%%%%%%%%%%%%%%%%%%%%%%%%%%%%%

In this paper we have presented a new, fast algorithm for solving Sorted L-One Penalized Estimation (SLOPE).
Our method relies on a combination of proximal gradient descent to identify the cluster structure of the solution and coordinate descent to allow the algorithm to take large steps.
In our results, we have shown that our method often outperforms all competitors by orders of magnitude for high-to-medium levels of regularization and typically performs among the best algorithms for low levels of regularization.

We have not, in this paper, considered using screening rules for SLOPE~\parencite{larsson2020c,elvira2022}.
Although screening rules work for any algorithm considered in this article, they are particularly effective when used in tandem with coordinate descent~\parencite{fercoq2015} and, in addition, easy to implement due to the nature of coordinate descent steps.
This point also holds to when fitting a path of \(\lambda\) sequences~\parencite{friedman2007,friedman2010}, which is often used during cross-validation to obtain an optimal \(\lambda\) sequence. 

Future research may consider alternative strategies to split clusters, for instance batch stochastic gradient descent, which should reduce the burden of the cluster splitting step.

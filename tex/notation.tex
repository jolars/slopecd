\paragraph{Notation}\label{sec:notation}

Let \((i)^{-}\) be the inverse of \((i)\) such that
\(\big((i)^-\big)^- = (i)\). See \cref{tab:permutation-example} for an
example of this operator for a particular \(\beta\).
\begin{table}
  \centering
  \caption{Example of the permutation operator \((i)\) and its inverse
    \((i)^-\)\label{tab:permutation-example}}
  \begin{tabular}{cccc}
    \toprule
    \(i\) & \(\beta_i\) & \((i)\) & \((i)^-\) \\
    \midrule
    1     & 0.5         & 2       & 3         \\
    2     & -5          & 3       & 1         \\
    3     & 4           & 1       & 2         \\
    \bottomrule
  \end{tabular}
\end{table}
This means that
\[
  J(\beta) = \sum_{j=1}^p \lambda_j |\beta_{(j)}|
  = \sum_{j=1}^p \lambda_{(j)^-}|\beta_j|.
\]
Sorted $\ell_1$ norm penalization produces vectors with clustered coefficients.
To this end, for a fixed $\beta$ such that $|\beta_j|$ takes $m$ values, we introduce \(\mathcal{C}_1, \mathcal{C}_2\, \dots, \mathcal{C}_m\) and \(c_1,
c_2, \dots, c_m\) for the indices and coefficients respectively of the \(m\)
clusters of $\beta$, such that
% \[
  $
  \mathcal{C}_i = \{j : |\beta_j| = c_i\} \text{ and }%\quad \text{and} \quad
  c_1 > c_2 > \cdots > c_m \geq 0.
  $
% \]
We also let \(\bar{\mathcal{C}}\) denote the complement of \(\mathcal{C}\).
Let $(e_i)_{i \in [d]}$ denote the canonical basis of $\bbR^d$.
Finally, let $\sign(x) = x / |x|$ (with the convention 0/0 = 1) be the scalar sign, that acts entrywise on vectors.
% \[
%   \sign(x) =
%   \begin{cases}
%     \phantom{-}1 & \text{if } x \geq 0 ,\\
%     -1           & \text{if } x < 0,
%   \end{cases}
% \]
% and allow \(\sign(x)\) to be overloaded for vectors.



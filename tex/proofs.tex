\section{Proofs}\label{sec:proofs}

\subsection{Proof of \Cref{thm:sl1-directional-derivative}}

\begin{proof}\label{app:proof_directional_derivative}
  \label{app:proof_directional_slope}
  From the observations in \Cref{rem:permutation_C_z},
  we have the following cases to consider: \(|z| \in c^{\setminus k}\),
  \(|z| = 0\), and \(|z| \notin \{0\} \cup c^{\setminus k}\).


  \paragraph{Case 1}
  Let us first consider the case $z \neq 0$, $|z| \neq c_i$ for all $i \neq k$.
  Since \(C(z + \delta h) = C(z) = \cC_k\) and $\sign(z + \delta h) = \sign(z)$ for $h$ small enough,
  \begin{align}
    H(z + \delta h) - H(z)
      &= \sum_{j =1}^p |\beta(z + \delta h)_j| \lambda_{(j)^-_{z + \delta h}}
          - \sum_{j=1}^p |\beta(z)_j| \lambda_{(j)^-_z} \nonumber \\
      &= \sum_{j =1}^p (|\beta(z + \delta h)_j| - |\beta(z)_j|) \lambda_{(j)^-_z} \nonumber  \\
      &= \sum_{j =1}^p (|\beta(z + \delta h)_j| - |\beta(z)_j|) \lambda_{(j)^-_z} \nonumber  \\
      &= \sum_{j \in C(z)}^p (|\beta(z + \delta h)_j| - |\beta(z)_j|) \lambda_{(j)^-_z} \nonumber \\
      &= \sum_{j \in C(z)} \sign(\beta(z)_j) (z + \delta h - z) \lambda_{(j)^-_z} \nonumber \\
      &= \sum_{j \in C(z)} \sign(z) \delta h  \lambda_{(j)^-_z} \nonumber \\
      &= \sum_{j \in \cC_k} \sign(z) \delta h  \lambda_{(j)^-_z} \, .
  \end{align}

  \paragraph{Case 2}
  Then if  $z \neq 0$ and $|z|$ is equal to one of the $c_i$'s, $i \neq k$,  one has $C(z) = \cC_k \cup \cC_i$, $C(z + \delta h) = \cC_k$, and $\sign(z + \delta h) = \sign(z)$ for $h$ small enough.
  Thus
  \begin{align}
    H(z + \delta h) - H(z)
      &= \sum_{j =1}^p |\beta(z + \delta h)_j| \lambda_{(j)^-_{z + \delta h}}
          - \sum_{i=1}^p |\beta(z)_j| \lambda_{(j)^-_z}  \nonumber \\
      &= \sum_{j \in \cC_k \cup \cC_i} \left( |\beta(z + \delta h)_j| \lambda_{(j)^-_{z + \delta h}}
         - |\beta(z)_j| \lambda_{(j)^-_z} \right)  \nonumber \\
      &= \sum_{j \in \cC_k} \left( c_i + \delta h \right) \lambda_{(i)^-_{z + \delta h}}
           - c_i \lambda_{(i)^-_z}
           + \sum_{j \in \cC_i} \left( c_i \lambda_{(j)^-_{z + \delta h}}
           - c_i \lambda_{(i)^-_z} \right) \, .
  \end{align}
  \mm{to conclude we need to mention there is an ambiguity in terms of permutation (the permutation reordering $\beta(z)$ is not unique, we can swap $\cC_i$ and $\cC_k$, but it does not change the value of the sum, so we can pick $(i)_z = (i)_{z + \delta_h}$ and conclude) as above.}

  \paragraph{Case 3} Finally let us treat the case $z = 0$.
  If $c_m = 0$ then the proof proceeds as in case 2, with the exception that $|\beta(z + \delta h)| = h$ and so the result is just:
  \begin{align}
    H(z + \delta h) - H(z)
      &= h \sum_{j \in \cC_k} \lambda_{(i)^-_{z + \delta h}} \, .
  \end{align}
  If $c_m \neq 0$, then the computation proceeds exactly as in case 1.
\end{proof}

\mathurin{previous proof below}
\subsection{Previous Proof of thm 2.2}
\begin{proof}\label{app:proof_directional_derivative}
    By definition of the directional derivative,
    \begin{align}
      H'(z; \delta)
      = & \lim_{h \downarrow 0} \frac{H(z + h \delta) - H(z)}{h} \nonumber \\
      = &
      \lim_{h \downarrow 0} \frac{1}{h}
      \left(
        |z + h \delta| \smashoperator{\sum_{j \in C(z + h \delta)}} \lambda_{(j)^-_{z + h \delta}}
        + \smashoperator{\sum_{j \notin C(z + h\delta)}} |\beta_j| \lambda_{(j)^-_{z + h \delta}}
        - |z| \smashoperator{\sum_{j \in C(z)}} \lambda_{(j)^-_z}
        - \smashoperator{\sum_{j \notin C(z)}} |\beta_j| \lambda_{(j)^-_z}
      \right) \nonumber                                                          \\
      = & \lim_{h \downarrow 0}
      \frac{1}{h}
      \left(
        |z + h \delta| \smashoperator{\sum_{j \in C(z + {\varepsilon_c}\delta)}} \lambda_{(j)^-_{z + {\varepsilon_c}\delta}}
        + \smashoperator{\sum_{j \notin C(z + {\varepsilon_c}\delta)}} |\beta_j| \lambda_{(j)^-_{z + {\varepsilon_c}\delta}}
        - |z| \smashoperator{\sum_{j \in C(z)}} \lambda_{(j)^-_{z}}
        - \smashoperator{\sum_{j \notin C(z)}} |\beta_j| \lambda_{(j)^-_{z}}
      \right)
      \label{eq:directional-derivative-sl1}
    \end{align}
    We have the following cases to consider: \(|z| \in c^{\setminus k}\),
    \(|z| = 0\), and \(|z| \notin \{0\} \cup c^{\setminus k}\).


    Let us first consider the case
    $z \neq 0$, $z \neq c_i$ for all $i \neq k$.
    Here since \(C(z + \delta h) = C(z)\) for $h$ small enough,
    %  has the directional derivative
    % \[
    %   H'(z; \delta) = \delta \sign(z) \smashoperator{\sum_{j \in C(z)}} \lambda_{(j)^-_{z}}.
    % \]

    Let us now treat the case \(|z| = c_i^{\setminus k}\), note that we have,
    as a result of the definition of
    \({\varepsilon_c}\), the following identities:
    \begin{align*}
      C(c_i^{\setminus k} + {\varepsilon_c}\delta)            & \subseteq C(c_i^{\setminus k}),                                                                                                                                 \\
      \tilde{\mathcal{C}}_i                           & = \widebar{C(c_i^{\setminus k} + {\varepsilon_c} \delta)} \cap C(c_i^{\setminus k}),                                                                                    \\
      C(c_i^{\setminus k})                                    & = \tilde{C}_i \cup \big(C(c_i^{\setminus k} + {\varepsilon_c}\delta) \cap C(c_i^{\setminus k})\big) = \tilde{\mathcal{C}}_i \cup C(c_i^{\setminus k} + {\varepsilon_c} \delta), \\
      \widebar{C(c_i^{\setminus k} + {\varepsilon_c} \delta)} & = \widebar{C(c_i^{\setminus k})} \cup \tilde{\mathcal{C}}_i.
    \end{align*}
    Using this, we can rewrite \eqref{eq:directional-derivative-sl1} as
    \begin{equation*}
      \label{eq:directional-derivative-simplified}
      H'(z; \delta)
      = \lim_{h \downarrow 0} \frac{1}{h}
      \left(
        \splitfrac{
        |z + h\delta|\smashoperator{\sum_{j \in C(z + {\varepsilon_c}\delta)}} \lambda_{(j)^-_{z + {\varepsilon_c}\delta}}
        + |c_i^{\setminus k}|\smashoperator{\sum_{j \in \tilde{C}_i}} \lambda_{(j)^-_{z + {\varepsilon_c}\delta}}
        + \smashoperator{\sum_{j \in \widebar{C(z)}}} |\beta_j|\lambda_{(j)^-_{z + {\varepsilon_c}\delta}}
        }{%
        - |z| \smashoperator{\sum_{j \in \tilde{\mathcal{C}}_i}} \lambda_{(j)^-_{z}}
        - |c_i^{\setminus k}| \smashoperator{\sum_{j \in C(z + {\varepsilon_c}\delta)}} \lambda_{(j)^-_{z}}
        - \smashoperator{\sum_{j \in \widebar{C(z)}}} |\beta_j| \lambda_{(j)^-_{z}}
        }
      \right).
    \end{equation*}
    Next, observe that \(\lambda_{(j)^-_{z + {\varepsilon_c}\delta}} =
    \lambda_{(j)^-_{z}}\) for all \(j \in \widebar{C(z)}\) and consequently
    \[
      \smashoperator{\sum_{j \in \widebar{C(z)}}} |\beta_j|\lambda_{(j)^-_{z + {\varepsilon_c}\delta}} =
      \smashoperator{\sum_{j \in \widebar{C(z)}}} |\beta_j|\lambda_{(j)^-_{z}}.
    \]
    Moreover, note that, since \(z = \pm c_i\), there exists a permutation corresponding to
    \(\lambda_{(j)^-_{z}}\) such that
    \[
      |z|\smashoperator{\sum_{j \in \tilde{C}_i}} \lambda_{(j)^-_{z + {\varepsilon_c}\delta}}
      = |c^{\setminus k}_i| \smashoperator{\sum_{j \in \tilde{C}_i}} \lambda_{(j)^-_{z}}
    \]
    and consequently
    \begin{equation}
      \label{eq:directional-derivative-simplified-again}
      H'(z; \delta)
      = \lim_{h \downarrow 0} \frac{1}{h}
      \left(
        |z + h\delta|\smashoperator{\sum_{j \in C(z + {\varepsilon_c}\delta)}} \lambda_{(j)^-_{z + {\varepsilon_c}\delta}}
        - |c_i^{\setminus k}| \smashoperator{\sum_{j \in C(z + {\varepsilon_c}\delta)}} \lambda_{(j)^-_{z}}.
      \right).
    \end{equation}
    Now, since \(c_i^{\setminus k} + h \delta > 0\) and \(-c_i^{\setminus k} + h \delta < 0\) in the limit as
    \(h\) goes to \(0\) for \(c_i \neq 0\), we have
    \[
      \lim_{h\downarrow 0} |-c_i + h \delta|
      = \lim_{h\downarrow 0}\big( |c_i^{\setminus k}| -h \delta\big)
      \quad\text{and}\quad
      \lim_{h\downarrow 0} |c_i^{\setminus k} + h \delta|
      = \lim_{h\downarrow 0}(|c_i^{\setminus k}| + h \delta)
    \]
    which means that
    \begin{align*}
      H'(z; \delta)
       & = \lim_{h \downarrow 0} \frac{1}{h}
      \left(
        \big(|z| + \sign(z)h\delta\big)\smashoperator{\sum_{j \in C(z + {\varepsilon_c}\delta)}} \lambda_{(j)^-_{z + {\varepsilon_c}\delta}}
        - |c_i| \smashoperator{\sum_{j \in C(z + {\varepsilon_c}\delta)}} \lambda_{(j)^-_{z}}.
      \right)                                                                                                                  \\
       & = \lim_{h \downarrow 0} \frac{1}{h}
      \sign(z)h\delta\smashoperator{\sum_{j \in C(z + {\varepsilon_c}\delta)}} \lambda_{(j)^-_{z + {\varepsilon_c}\delta}}     \\
       & = \sign(z)\delta\smashoperator{\sum_{j \in C(z + {\varepsilon_c}\delta)}} \lambda_{(j)^-_{z + {\varepsilon_c}\delta}} \\
    \end{align*}

    For the case when \(z=0\) there are two sub-cases to consider. First, when \(c_m^{\setminus k} = 0\) then
    \eqref{eq:directional-derivative-simplified-again} is simply
    \begin{equation*}
      \label{eq:directional-derivative-zerocase}
      H'(0; \delta)
      = \lim_{h \downarrow 0} \frac{1}{h}
      |h\delta|\smashoperator{\sum_{j \in C({\varepsilon_c}\delta)}} \lambda_{(j)^-_{{\varepsilon_c}\delta}}
      = \smashoperator{\sum_{j \in C({\varepsilon_c}\delta)}} \lambda_{(j)^-_{{\varepsilon_c}\delta}}
    \end{equation*}
    since \(|\delta| = 1\) by definition.

    The other sub-case is \(c_m^{\setminus k} \neq 0\). Here
    \eqref{eq:directional-derivative-sl1} reduces to
    \begin{equation*}
      H'(0; \delta) = \lim_{h \downarrow 0}
      \frac{1}{h}
      \left(
        |h \delta| \smashoperator{\sum_{j \in C({\varepsilon_c}\delta)}} \lambda_{(j)^-_{{\varepsilon_c}\delta}}
        + \smashoperator{\sum_{j \notin C({\varepsilon_c}\delta)}} |\beta_j| \lambda_{(j)^-_{{\varepsilon_c}\delta}}
        - \smashoperator{\sum_{j \notin C(0)}} |\beta_j| \lambda_{(j)^-_{0}}
      \right).
    \end{equation*}
    In this case, we have \(C(0) = C({\varepsilon_c}\delta)\) since \(c_m^{\setminus k}
    \neq 0\), and therefore
    \[
      \smashoperator{\sum_{j \notin C({\varepsilon_c}\delta)}} |\beta_j| \lambda_{(j)^-_{{\varepsilon_c}\delta}}
      = \smashoperator{\sum_{j \notin C(0)}} |\beta_j| \lambda_{(j)^-_{0}},
    \]
    which means that
    \begin{equation*}
      H'(0; \delta) = \lim_{h \downarrow 0}
      \frac{1}{h}
      |h \delta| \smashoperator{\sum_{j \in C({\varepsilon_c}\delta)}} \lambda_{(j)^-_{{\varepsilon_c}\delta}}
      = |\delta|\smashoperator{\sum_{j \in C({\varepsilon_c}\delta)}} \lambda_{(j)^-_{{\varepsilon_c}\delta}}
      = \smashoperator{\sum_{j \in C(0)}} \lambda_{(j)^-_{0}}.
    \end{equation*}
  \end{proof}


\subsection{Proof of \Cref{thm:thresholding-operator}}

\begin{proof}
  Recall that \(G(z) : \mathbb{R} \mapsto \mathbb{R}\) is a convex,
  continuous piecewise-differentiable function with kinks whenever \(|z| =
  c_i^{\setminus k}\) or \(z = 0\). Let \(\gamma = \tilde{r}^T\tilde{x}\)
  and \(\omega = \tilde{x}^T\tilde{x}\) and note that the optimality criterion for
  \eqref{pb:cluster-problem} is
  \[
    \delta(\omega z - \gamma) + H'(z; \delta) \geq 0, \quad
    \forall \delta \in \{-1, 1\},
  \]
  which is equivalent to
  \begin{equation}
    \label{eq:optimality-inequality}
    \omega z - H'(z; -1) \leq \gamma \leq \omega z + H'(z; 1).
  \end{equation}
  We now proceed to show that there is a solution \(z^* \in \argmin_{z \in
    \mathbb{R}} H(z)\) for every interval over \(\gamma \in \mathbb{R}\).

  First, assume that the first case in the definition of \(T_k\) holds
  and note that this is equivalent to \eqref{eq:optimality-inequality} with \(z
  = 0\) since \(C({\varepsilon_c}) = C(-{\varepsilon_c})\) and
  \(\lambda_{(j)^-_{-{\varepsilon_c}}} = \lambda_{(j)^-_{{\varepsilon_c}}}\).
  This is sufficient for \(z^* = 0\).

  Next, assume that the second case holds and observe that this is equivalent
  to \eqref{eq:optimality-inequality} with
  \(z = c_i^{\setminus k}\), since
  \(C(c_i + {\varepsilon_c}) = C(-c_i - {\varepsilon_c})\) and
  \(C(-c_i + {\varepsilon_c}) = C(c_i - {\varepsilon_c})\). Thus \(z^* =
  \sign(\gamma)c_i^{\setminus k}\).

  For the third case, we have
  \[
    \smashoperator{\sum_{j \in C(c_i + {\varepsilon_c})}} \lambda_{(j)^-_{c_i + {\varepsilon_c}}}
    =
    \smashoperator[r]{\sum_{j \in C(c_{i-1} - {\varepsilon_c})}} \lambda_{(j)^-_{c_{i-1} - {\varepsilon_c}}}
  \]
  and therefore \eqref{eq:optimality-inequality} is equivalent to
  \[
    c_i < \frac{1}{\omega} \bigg( |\gamma| - \smashoperator{\sum_{j \in C(c_i + {\varepsilon_c})}} \lambda_{(j)^-_{c_i + {\varepsilon_c}}} \bigg) < c_{i -1}.
  \]
  Now let
  \begin{equation}
    \label{eq:differentiable-solution}
    z^* = \frac{\sign(\gamma)}{\omega} \bigg( |\gamma| - \smashoperator{\sum_{j \in C(c_i + {\varepsilon_c})}} \lambda_{(j)^-_{c_i + {\varepsilon_c}}} \bigg)
  \end{equation}
  and note that \(|z^*| \in \big(c_i^{\setminus k}, c_{i-1}^{\setminus k}\big)\) and hence
  \[
    \frac{1}{\omega} \bigg( |\gamma| - \smashoperator{\sum_{j \in C(c_i + {\varepsilon_c})}} \lambda_{(j)^-_{c_i + {\varepsilon_c}}} \bigg)
    =
    \frac{1}{\omega} \bigg( |\gamma| - \smashoperator{\sum_{j \in C(z^*)}} \lambda_{(j)^-_{z^*}} \bigg).
  \]
  Furthermore, since \(G\) is differentiable in \(\big(c_i^{\setminus k}, c_{i-1}^{\setminus k}\big)\), we have
  \[
    \frac{\partial}{\partial z} G(z) \Big|_{z = z^*}
    = \omega z^* - \gamma + \sign(z^*) \smashoperator{\sum_{j \in C(z^*)}} \lambda_{(j)^-_{z^*}} = 0,
  \]
  and therefore \eqref{eq:differentiable-solution} must be the solution.

  The solution for the last case follows using reasoning analogous to that of the
  third case.
\end{proof}

\subsection{Proof of \Cref{lem:convergence}}

\begin{proof}
  From \cref{thm:thresholding-operator}, we know that lines 9--14 in \cref{alg:hybrid} correspond to minimizing \(G(z)\) for a given \(\beta \coloneqq \beta^{(t + k / |\mathcal{C}|)}\), and therefore that
  \[
    G\big(\beta^{(t + (k - 1) / |\mathcal{C}|)}\big) \leq G\big(\beta^{(t + k / |\mathcal{C}|)}\big)
  \]
  since \(G(z) = P\big(\beta(z)\big)\).

  Moreover, for a sequence of iterates of the proximal gradient descent step,
  \(\beta^{(v)}, \beta^{(2v)}, \dots, \beta^{(\lfloor k / v \rfloor)}\),
  we know from~\textcite[Theorem 3.1]{beck2009} that it holds that
  \[
    P(\beta^{(\lfloor k / v \rfloor)}) - P(\beta^*)
    \leq \frac{L \lVert \beta^{(0)} - \beta^* \rVert_2^2}{2\lfloor k / v \rfloor}.
  \]
  Combining this and the result from the previous paragraph yields the desired
  result.
\end{proof}

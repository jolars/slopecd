\section{Experiments}\label{sec:experiments}
%%%%%%%%%%%%%%%%%%%%%%%%%%%%%%%%%%%%%%%%%%%

The proposed algorithm is part of a python package relying on numpy and numba.
It will made open-source upon publication. To compare its efficiency, we used several public datasets described in \cref{table:datasets}.
We performed an extensive benchmark with the following competitors:
\begin{itemize}[noitemsep]
    \item Alterning Direction Method of Multipliers (\texttt{admm}) \cite{boyd2010}
    \item Anderson acceleration for proximal gradient descent (\texttt{anderson}) \cite{zhang2020}
    \item Proximal Gradient Descent (\texttt{pgd}) \cite{combettes2005}
    \item Fast Iterative Shrinkage-Thresholding Algorithm (\texttt{fista}) \cite{beck2009}
    \item Semismooth Newton-Based Augmented Lagrangian (\texttt{newt-alt}) \cite{Ziyan2019}
    \item The Oracle solver (\texttt{oracle}) uses the clusters obtained via another
     solver to compute coordinate descent updates from the known solver.
    \item The Hybrid (ours) (\texttt{hybrid}) solver (see \cref{alg:hybrid}) combines proximal gradient descent
     and coordinate descent to overcome the non-separability of the SLOPE problem.
\end{itemize}

\begin{table}[]
    \centering
    \label{table:datasets}
    \begin{tabular}{cccc}
    \hline
    Datasets    & \#samples n & \#features p & density \\ \hline
    Rhee2006    & 842         & 361          & ?       \\ 
    bcTCGA      & 536         & \num{17322}        & 1       \\
    Scheetz2006 & 120         & \num{18975}        & ?       \\ \hline
    \end{tabular}
\end{table}

We used the \texttt{benchopt} \cite{moreau2022benchopt} tool to obtain the convergence curves for the different solvers.
\texttt{Benchopt} launches each solver several times increasing the number of iterations and store the objective value, dual gap and time to reach it.
The repository to reproduce the benchmark is available at XXX.

\subsection{Benchmark on simulated data}

\subsection{Benchmark on real datasets}
\begin{figure}[htbp]
    \centering
    \includegraphics[scale=0.47]{Rhee2006_legend.pdf}
    \includegraphics[scale=0.5]{Rhee2006.pdf}
    \caption{Benchmark on the Rhee2006 dataset.}
    \label{fig:Rhee2006}
  \end{figure}
\section{COMPUTATIONAL DETAILS}

\label{sec:computational-details}

Here we describe some computational details of the coordinate update step of our algorithm.

\subsection{Naive Updates}

As in \textcite{friedman2010}, we can improve the efficiency of updates by observing that
\begin{equation*}
  \begin{aligned}
    \tilde r_k & = y - \tilde y_k                                                                       \\
               & = y - X_{\bar{\mathcal{C}}_k}\beta_{\bar{\mathcal{C}}_k} - \tilde x c_k + \tilde x c_k \\
               & = r + \tilde x c_k
  \end{aligned}
\end{equation*}
and therefore that
\begin{equation}
  \label{eq:naive-update}
  \tilde x_k^T (y - \tilde y_k) = \tilde x_k^T r + \tilde x_k^T \tilde x_k c_k.
\end{equation}

\subsection{Caching Reductions}

Observe that \(\tilde x_k\) only changes between subsequent coordinate updates provided that the members of the cluster \(k\) change, for instance if two clusters are merged, a predictor leaves a cluster, or the signs flip (through an update of \(\alpha_k\)).
As a result, it is possible to obtain computational gains by caching \(\tilde x_k\) and \(\tilde x_k^T \tilde x_k\) for each cluster (except the zero cluster, which we do not consider in our coordinate descent step).
When there is no change in the clusters, there is no need to recompute these quantities.
And even when there are changes, we can still reduce the costs involved since \(\tilde x_k\) can be updated in place.
If a large cluster is joined by few new predictors, then the cost of updating may be much lower than recomputing the quantities for the entire cluster.
Also note that, for single-member clusters we only need to store \(\tilde x_k^T \tilde x_k\) since \(\tilde x_k\) is just a column in \(X\) times the corresponding sign.

Letting \(\tilde x_k^\text{old}\) correspond to the value of \(\tilde x_k\) before the update, we note that \(\tilde x_k \gets \tilde x_k^\text{old} + x_j \sign(\beta_j)\) for each \(j \in \mathcal{C}_k^\text{new} \setminus \mathcal{C}_k^\text{old}\) and \(\tilde x_k \gets \tilde x_k^\text{old} - x_j \sign(\beta_j)\) for each \(j \in \mathcal{C}_k^\text{old} \setminus \mathcal{C}_k^\text{new}\).
If only the signs flip, we simply have to also flip the signs in \(\tilde x_k\).

In practice, however, we have so far not found that this type of caching leads to substantial improvements in running time.
As a result, we have not included them in the experiments in this paper.

\subsection{Covariance Updates}

Notice that we can rewrite the first term in \eqref{eq:naive-update} as
\begin{equation}
  \begin{aligned}
    \tilde x_k^T r & = \tilde x_k^T y - \sum_{j : \beta_j \neq 0} \tilde x_k^T x_j \beta_j                                                                    \\
                   & = \tilde x_k^T y - \sum_{j : c_j \neq 0} \tilde x_k^T \tilde x_j c_j                                                                     \\
                   & = s_{\mathcal{C}_k}^T X_{\mathcal{C}_k}^T y - \sum_{j : \beta_j \neq 0} s_{\mathcal{C}_k}^T X_{\mathcal{C}_k}^T x_j \beta_j              \\
                   & = s_{\mathcal{C}_k}^T \left(X^T y\right)_{\mathcal{C}_k} - \sum_{j : \beta_j \neq 0} s_{\mathcal{C}_k}^T X_{\mathcal{C}_k}^T x_j \beta_j \\
                   & = \sum_{j \in \mathcal{C}_k}\left( s_j x_j^Ty - \sum_{t : \beta_t \neq 0} s_j x_j^T x_t \beta_t \right)
  \end{aligned}
\end{equation}
As in \textcite{friedman2010}, this formulation can be used to achieve so-called \emph{covariance updates}.
We compute \(X^T y\) once at the start.
Then, each time a new predictor becomes non-zero, we compute its inner product with all other predictors, caching these products.
Note that we do not use these updates in our experiments since they are useful only for the ordinary SLOPE case and would limit the generalizability of the results.

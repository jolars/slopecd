\subsection{Notation}\label{sec:notation}

Let \((i)^{-}\) be the inverse of \((i)\) such that
\(\big((i)\big)^- = i\).
\begin{example}
  \begin{tabular}{cccc}
    \toprule
    \(i\) & \(\beta\) & \((i)\) & \((i)^-\) \\
    \midrule
    1     & 1         & 2       & 3         \\
    2     & -3        & 3       & 1         \\
    3     & 2         & 1       & 2         \\
    \bottomrule
  \end{tabular}
\end{example}
Note that this means that
\[
  J(\beta) = \sum_{j=1}^p \lambda_j |\beta_{(j)}|
           = \sum_{j=1}^p \lambda_{(j)^-}|\beta_j| \enspace.
\]
%
Furthermore for $\mathcal{A} \subset [p]$, let
\[
  S(\mathcal{A}) = \sum_{i \in \mathcal{A}} \lambda_{(i)^-} \enspace.
\]
%
A key property of SLOPE is that the coefficients of $\beta$ tend to cluster.
Let \(\mathcal{C}_1, \mathcal{C}_2\, \dots, \mathcal{C}_m\) be the
clusters such that\mm{this notation may be ambiguous as $m$ depends on $\beta$}
\[
  \mathcal{C}_i = \{j : |\beta_j| = c_i\} \quad \text{and} \quad
  c_1 > c_2 > \cdots > c_m \geq 0.
\]
In addition, let \(C(\alpha)\) be a function that returns the
cluster corresponding to \(\alpha\), that is
\[
  C(\alpha) = \{j : |\beta_j| = \alpha\}.
\]
Let \(\lambda^{\mathcal{A}}\) be the sub-sequence of
\(\lambda\) corresponding to the \(\mathcal{A}\) cluster, i.e.
\[
  \lambda^\mathcal{A} = \{\lambda_{(i)^-} : i \in \mathcal{A}\}
\]\mathurin{put on the same level as $\mathcal{S}(\mathcal{A})$ definition?}





\subsection{Notation}\label{sec:notation}

Let \((i;\beta)^{-}\) be the inverse of \((i;\beta)\) such that
\(\big((i;\beta)^-;\beta\big)^- = (i;\beta)\). See \cref{tab:permutation-example} for an
example of this operator for a particular \(\beta\).
\begin{table}
  \centering
  \caption{Example of the permutation operator \((i)\) and its inverse
    \((i)^-\)\label{tab:permutation-example}}
  \begin{tabular}{cccc}
    \toprule
    \(i\) & \(\beta\) & \((i; \beta)\) & \((i; \beta)^-\) \\
    \midrule
    1     & 1         & 2              & 3                \\
    2     & -3        & 3              & 1                \\
    3     & 2         & 1              & 2                \\
    \bottomrule
  \end{tabular}
\end{table}
This means that
\[
  J(\beta) = \sum_{j=1}^p \lambda_j |\beta_{(j;\beta)}|
  = \sum_{j=1}^p \lambda_{(j;\beta)^-}|\beta_j|.
\]
Also, let \(\mathcal{C}_1, \mathcal{C}_2\, \dots, \mathcal{C}_m\) and \(c_1,
c_2, \dots, c_m\) be the indices and coefficients, respectively, for the \(m\)
clusters such that
\[
  \mathcal{C}_i = \{j : |\beta_j| = c_i\} \quad \text{and} \quad
  c_1 > c_2 > \cdots \geq 0.
\]
In addition, let \(C(\alpha)\) be a function that returns the
cluster corresponding to \(\alpha\), that is
\[
  C(\alpha) = \{j : |\beta_j| = \alpha\}.
\]
Let \(\lambda^{\mathcal{A}}\) be the sub-sequence of
\(\lambda\) corresponding to the \(\mathcal{A}\) cluster, i.e.
\[
  \lambda^\mathcal{A} = \{\lambda_{(i)^-} : i \in \mathcal{A}\}
\]





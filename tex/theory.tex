\subsection{Directional Derivatives}%
\label{sec:directional-derivatives}
\subsubsection{The Sorted \texorpdfstring{\(\ell_1\)}{l1}
  Norm}

\begin{theorem}
  \label{thm:sl1-directional-derivative}
  Let \(h_0 \in \big(0, \min_{i,j \in \{i \mid \beta_i \neq 0\}}\big| |\beta_i| -
  |\beta_j| \big| \big]\) and define \((\cdot)^*\) to be the
  permutation such that
  \[
    |\beta + h_0v|_{(1)^*} \geq |\beta + h_0v |_{(2)^*}
    \geq \cdots \geq |\beta + h_0v|_{(p)^*}.
  \]
  The directional derivative for the sorted \(\ell_1\) norm, \(J(\beta,
  \lambda)\), is
  \[
    D_v J(\beta ; \lambda) =
    \sum_i \sum_{j \in \mathcal{B}_i} \lambda_j v_{(j)^*}\sign(\beta_{(j)^*} + h_0v_{(j)^*})\]
  where
  \[
    \mathcal{B}_i = \{j \mid |\beta_j| = |\beta_i|\}.
  \]
  \todo[author=JL]{We need something that doesn't define clusters multiple times.}
\end{theorem}
\begin{proof}
  The directional derivative for the sorted \(\ell_1\) norm and a direction
  \(v\) with \(\lVert v \rVert = 1\) is
  \begin{equation}
    \label{eq:sl1-directional-derivative}
    \begin{aligned}
      D_v J(\beta, \lambda) & = \lim_{h \searrow 0} \frac{J(\beta + h v; \lambda) - J(\beta; \lambda)}{h}                                            \\
                            & = \lim_{h \searrow 0} \frac{\sum_{j=1}^p\lambda_j\big(|\beta + vh|_{(j)} - |\beta|_{(j)}\big)}{h}                      \\
                            & = \lim_{h \searrow 0}\frac{\sum_i \sum_{j \in \mathcal{B}_i} \lambda_j\big(|\beta + vh|_{(j)} - |\beta|_{(j)}\big)}{h} \\
    \end{aligned}
  \end{equation}
  First, assume that that is an \(i\) such that \(\card \mathcal{B}_i \neq 0\)
  and \(\beta_i = 0\).
  Then
  \[
    \begin{aligned}
      \sum_{j \in \mathcal{B}_i}\frac{\lambda_j \big( |\beta + vh|_{(j)} - |\beta|_{(j)}\big)}{h}
       & = \sum_{j \in \mathcal{B}_i} \lambda_j \sign(v)_{(j)}v_{(j)}               \\
       & = \sum_{j \in \mathcal{B}_i} \lambda_j \sign(\beta + hv)_{(j)^*}v_{(j)^*},
    \end{aligned}
  \]
  where the last equality follows from the fact that \(\sign(\beta + hv) =
  \sign(\beta)\) and \((i) = (i)^*\) for all \(i\) whenever \(h > 0\) and
  \(\beta \neq 0\).

  Next, for each \(i\) such that \(\card \mathcal{B}_i \neq 0\) and
  \(\beta_i \neq 0\), observe that \(\sign(\beta + hv) = \sign(\beta)\) and
  \((i)^* = (i)\) whenever \(0 < h < h_0\), recalling the construction
  of \(h_0\).
  Therefore, we have
  \[
    \sum_{j \in \mathcal{B}_i} \frac{\lambda_j\big(|\beta + hv|_{(j)} - |\beta|_{(j)}\big)}{h}
    = \sum_{j \in \mathcal{B}_i} \lambda_j\sign(\beta + vh)_{(j)^*}v_{(j)^*}.
  \]
  From this, we see that \eqref{eq:sl1-directional-derivative} reduces to
  \[
    \lim_{h \searrow 0} \sum_i \sum_{j \in \mathcal{B}_i} \lambda_j\sign(\beta + vh)_{(j)^*}v_{(j)^*}
    = \sum_i \sum_{j \in \mathcal{B}_i} \lambda_j\sign(\beta + vh_0)_{(j)^*}v_{(j)^*}.
  \]
\end{proof}

\begin{remark}
  The directional derivative for \eqref{eq:slope-problem} is
  \[
    D_v f(\beta; \lambda) = v^T \big(\nabla f(\beta; \lambda)\big) + D_v J(\beta; \lambda).
  \]
\end{remark}

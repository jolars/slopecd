\paragraph{Notation}\label{sec:notation}

Let \((i)^{-}\) be the inverse of \((i)\) such that
\(\big((i)^-\big)^- = (i)\); see \Cref{tab:permutation-example} for an
example of this operator for a particular \(\beta\).
\begin{table}[bt]
  \centering
  \caption{Example of the permutation operator \((i)\) and its inverse
    \((i)^-\) for $\beta = [0.5, -5, 4]^T$}
    \label{tab:permutation-example}
  \begin{tabular}{cS[table-format=-1.1,round-mode=off]cc}
    \toprule
    \(i\) & {\(\beta_i\)} & \((i)\) & \((i)^-\) \\
    \midrule
    1     & 0.5         & 2       & 3         \\
    2     & -5          & 3       & 1         \\
    3     & 4           & 1       & 2         \\
    \bottomrule
  \end{tabular}
\end{table}
This means that
\[
  J(\beta) = \sum_{j=1}^p \lambda_j |\beta_{(j)}|
  = \sum_{j=1}^p \lambda_{(j)^-}|\beta_j| \,.
\]

Sorted $\ell_1$ norm penalization leads to solution vectors with clustered coefficients in which the absolute values of several coefficients are set to exactly the same value.
To this end, for a fixed $\beta$ such that $|\beta_j|$ takes $m$ distinct absolute values, we introduce \(\mathcal{C}_1, \mathcal{C}_2\, \dots, \mathcal{C}_m\) and \(c_1,
c_2, \dots, c_m\) for the indices and coefficients respectively of the \(m\)
clusters of $\beta$, such that
$\mathcal{C}_i = \{j : |\beta_j| = c_i\}$ and $c_1 > c_2 > \cdots > c_m \geq 0.$
Let \(\bar{\mathcal{C}}\) denote the complement of the set \(\mathcal{C}\).
Furthermore, let $(e_i)_{i \in [d]}$ denote the canonical basis of $\bbR^d$, with \([d] = \{1,2,\dots,d\}\).
Let $X_{i:}$ and $X_{:i}$ denote the $i$th row and column, respectively, of the matrix $X$.
Finally, let $\sign(x) = x / |x|$ (with the convention 0/0 = 1) be the scalar sign, that acts entrywise on vectors.



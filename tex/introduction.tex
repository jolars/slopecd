%!TEX root = ./slopecd.tex
\section{Introduction}\label{sec:introduction}
%%%%%%%%%%%%%%%%%%%%%%%%%%%%%%%%%%%%%%%%%%%%%%

In this paper we present a novel numerical algorithm for Sorted L-One Penalized
Estimation (SLOPE)~\cite{bogdan2013, bogdan2015}, defined as
\begin{problem}\label{pb:slope}
  \min_{\beta \in \mathbb{R}^p}
  P(\beta) = L(\beta) + J(\beta)
\end{problem}
% \mm{TODO change}
where we take \(L\) to be smooth and twice differentiable and
% \mm{to discuss: $L(X\beta)$ doable? + some computation are sepcific to  the quadratic case}
\begin{equation}
  \label{eq:sorted-l1-norm}
  J(\beta) = \sum_{j=1}^p \lambda_j|\beta_{(j)}|
\end{equation}
is the \emph{sorted \(\ell_1\) norm}~\cite{zeng2014ordered}, defined through
\begin{equation*}
  |\beta_{(1)}| \geq |\beta_{(2)}| \geq \cdots \geq |\beta_{(p)}| \enspace,
\end{equation*}
with \(\lambda\) a fixed non-increasing and non-negative
sequence.

The sorted $\ell_1$ norm is a sparsity-enforcing penalty  that has become increasingly popular due to
several appealing properties, such as its ability to control false discovery
rate~\cite{bogdan2015, kos2020}, cluster coefficients~\cite{figueiredo2016,
  schneider2020a}, and recover sparsity and ordering patterns in the
solution~\cite{bogdan2022}. Unlike other competing sparse regularization methods such
as MCP~\cite{zhang2010} and SCAD~\cite{fan2001}, SLOPE is also a convex
problem~\cite{bogdan2015}.

In spite of the availability of predictor screening rules~\cite{larsson2020c,elvira2022}, which help speed up SLOPE in the high-dimensional regime,
current state-of-the-art algorithms for SLOPE perform poorly in comparison to
those of more established penalization methods such as the lasso
(\(\ell_1\)-norm regularization) and ridge regression (\(\ell_2\)-norm
regularization). As a small illustration of this issue, we compared the
speed at which the SLOPE and glmnet packages fit a complete regularization
path for the bcTCGA data set. SLOPE takes x seconds to fit the full path,
whilst glmnet requires only y seconds. \mm{put result here and details in experiment section}




This lackluster performance has hampered the applicability of SLOPE to many
real-world applications. % particularly because users often need to solve SLOPE
%repeatedly when tuning hyper-parameters, for instance in
%cross-validation, or when fitting SLOPE as part of Adaptive Bayesian
% SLOPE~\cite{jiang2022}.
A major reason for why algorithms for solving $\ell_1$-, MCP-, or SCAD-regularized problems enjoy better
performance is that they use coordinate
descent~\cite{tseng2001convergence,friedman2010,breheny2011}.
On the other hand, current SLOPE solvers rely on proximal gradient descent
algorithms such as FISTA~\cite{beck2009} and the
alternating direction method of multipliers method (ADMM)~\cite{boyd2010}, that in ML scenario are slower than coordinate descent \cite{moreau2022benchopt}.
% which are both used in the current version of the SLOPE package.
Applying coordinate descent to SLOPE is, however, not straightforward
because convergence guarantees for coordinate descent require the
objective to be separable, which %, as we can see from \eqref{eq:sorted-l1-norm}
is not the case for SLOPE.

% The resulting solution \(\hat{\beta}\) has the property that it can cluster the
% coefficients by their magnitudes, such that \(\mathcal{C}=
% \{j: |\beta_j|=c\}\) for some \(c\).

In this article we address this problem by introducing a new, highly effective
algorithm for SLOPE based on a hybrid proximal gradient and coordinate descent
scheme. Our method features convergence guarantees and reduces the time
required to fit SLOPE by orders of magnitude in our empirical experiments.

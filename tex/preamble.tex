
\usepackage{lmodern}
\usepackage{amssymb,amsmath,amsthm,mathtools,isomath}

\usepackage[T1]{fontenc}
\usepackage[utf8]{inputenc}
\usepackage{textcomp} % provide euro and other symbols
\usepackage[english]{babel}

\usepackage{upquote} % straight quotes in verbatim environments
\usepackage{nicefrac}	% compact symbols for 1/2, etc.
\usepackage[]{microtype}
\UseMicrotypeSet[protrusion]{basicmath} % disable protrusion for tt fonts

\usepackage{stmaryrd}
\usepackage{xcolor}
\usepackage{xurl}
\usepackage{bookmark}
\usepackage{csquotes}
\usepackage{siunitx}
\usepackage{enumitem}

\usepackage{algorithm,algpseudocode}
\usepackage[titlenumbered,linesnumbered,ruled,noend,algo2e]{algorithm2e}

\usepackage[textsize=scriptsize]{todonotes}
\newcommand{\jonas}[1]{\textcolor{red!20}{[JW: #1]}}
\newcommand{\jw}[1]{\todo[inline,color=red!20]{{\bf JW:} #1}}
\newcommand{\mm}[1]{\todo[inline,color=purple!20]{{\bf Mathurin:} #1}}
\newcommand{\mathurin}[1]{\todo[inline,color=purple!20]{{\bf Mathurin:} #1}}
\newcommand{\klopfe}[1]{\todo[inline,color=orange]{{\bf Klopfe:} #1}}
\newcommand{\qk}[1]{\todo[inline,color=orange]{{\bf Klopfe:} #1}}
\newcommand{\johan}[1]{\todo[inline,color=black!30]{{\bf JL:} #1}}
\newcommand{\jl}[1]{\todo[inline,color=black!30]{{\bf JL:} #1}}
\newcommand{\JL}[1]{{\todo[inline,color=black!30]{{\bf JL:} #1}}}

\newcommand{\widebar}[1]{\mkern 1.5mu\overline{\mkern-1.5mu#1\mkern-1.5mu}\mkern 1.5mu}

\usepackage[noabbrev,nameinlink]{cleveref}

%%%% problem environment
\usepackage{aliascnt}
\newaliascnt{problem}{equation}
\aliascntresetthe{problem}
\creflabelformat{problem}{#2\textup{(#1)}#3}
\makeatletter
\def\problem{$$\refstepcounter{problem}}
\def\endproblem{\eqno \hbox{\@eqnnum}$$\@ignoretrue}
\makeatother
\Crefname{problem}{Problem}{Problems}
%%%%%%%%%%%%%%%%%

\usepackage{hyperref}
\hypersetup{
  colorlinks = true,
  linkcolor  = RoyalBlue4,
  filecolor  = RoyalBlue4,
  citecolor  = VioletRed4,
  urlcolor   = RoyalBlue4
}

\usepackage{longtable}
\usepackage{booktabs}
\usepackage{csvsimple}

\usepackage{etoolbox}

% Allow footnotes in longtable head/foot
\usepackage{footnotehyper}
\makesavenoteenv{longtable}

\usepackage{graphicx}
\graphicspath{{figures/}}
\usepackage{subcaption}

% bibliography
\usepackage[style=authoryear,language=english,uniquelist=false,maxbibnames=10,minbibnames=5,uniquename=false,doi=false]{biblatex}
\addbibresource{slopecd.bib}
\DeclareDelimFormat{nameyeardelim}{\addcomma\space}

\setlength{\emergencystretch}{3em} % prevent overfull lines

% operators
\DeclareMathOperator*{\argmax}{arg\,max}
\DeclareMathOperator*{\argmin}{arg\,min}
\DeclareMathOperator{\tr}{tr}
\DeclareMathOperator{\diag}{diag}
\DeclareMathOperator{\range}{range}
\DeclareMathOperator{\nullspace}{null}
\DeclareMathOperator{\rank}{rank}
\DeclareMathOperator{\sign}{sign}
\DeclareMathOperator{\dist}{dist}
\DeclareMathOperator{\prox}{prox}
\DeclareMathOperator{\Span}{span}
\DeclareMathOperator{\cumsum}{cumsum}
\DeclareMathOperator{\parset}{par}

% delimiters
\DeclarePairedDelimiter\ceil{\lceil}{\rceil}
\DeclarePairedDelimiter\floor{\lfloor}{\rfloor}

% macros
\newcommand{\abs}[1]{\lvert {#1} \rvert}
\newcommand{\norm}[1]{\lVert {#1} \rVert}
\newcommand{\bbR}{\mathbb{R}}
\newcommand{\cB}{\mathcal{B}}
\newcommand{\cC}{\mathcal{C}}
\newcommand{\cG}{\mathcal{G}}
\newcommand{\cM}{\mathcal{M}}
\newcommand{\pkg}[1]{\textsf{#1}}
\newcommand{\dataset}[1]{\texttt{#1}}

% theorems
\theoremstyle{plain}
\newtheorem{theorem}{Theorem}[section]
\newtheorem{proposition}[theorem]{Proposition}
\newtheorem{lemma}[theorem]{Lemma}
\newtheorem{corollary}[theorem]{Corollary}
\theoremstyle{definition}
\newtheorem{definition}[theorem]{Definition}
\newtheorem{assumption}[theorem]{Assumption}
\theoremstyle{remark}
\newtheorem{remark}[theorem]{Remark}
\newtheorem{example}{Example}
